%%%%%%%%%%%%%%%%%%%%%%%%%%%%%%%%%%%%%%%%%
% Medium Length Professional CV
% LaTeX Template
% Version 2.0 (8/5/13)
%
% This template has been downloaded from:
% http://www.LaTeXTemplates.com
%
% Original author:
% Rishi Shah 
%
% Important note:
% This template requires the resume.cls file to be in the same directory as the
% .tex file. The resume.cls file provides the resume style used for structuring the
% document.
%
%%%%%%%%%%%%%%%%%%%%%%%%%%%%%%%%%%%%%%%%%

%----------------------------------------------------------------------------------------
%	PACKAGES AND OTHER DOCUMENT CONFIGURATIONS
%----------------------------------------------------------------------------------------

\documentclass[a4paper]{resume} % Use the custom resume.cls style

\usepackage[left=0.75in,top=0.6in,right=0.75in,bottom=0.6in]{geometry} % Document margins
\newcommand{\tab}[1]{\hspace{.2667\textwidth}\rlap{#1}}
\newcommand{\itab}[1]{\hspace{0em}\rlap{#1}}
\name{Christoph Oliver Roupec} % Your name
\address{Breitenseer Straße 39a, T23-24, 1140 Vienna} % Your address
%\address{123 Pleasant Lane \\ City, State 12345} % Your secondary addess (optional)
\address{(+43) 664 3826128 \\ Christoph.Roupec@gmail.com} % Your phone number and email

\begin{document}

%----------------------------------------------------------------------------------------
%	Education and Experience
%----------------------------------------------------------------------------------------

\begin{rSection}{Experience}

{\bf Vienna University of Technology, Institute for Theoretical Physics, Vienna} \hfill {\em 2017 - 2022} 
\\ Doctorate in Theoretical Physics under the supervision of {\bf Timm Wrase}.
\\ Thesis Title: \emph{Flux Compactifications, dS Vacua and the Swampland}\vspace{7pt}
\\{\bf Stanford Institute for Theoretical Physics} \hfill {\em August - December 2019}
\\ Visiting Scientist working with {\bf Renata Kallosh} and {\bf Andrei Linde}.
%\\ Uplifting $\overline{D6}$-branes
\vspace{7pt}
\\{\bf University of Vienna, Vienna} \hfill {\em 2015 - 2017} 
\\ Master of Physics\hfill {graduated with distinction }
\\ Emphasis on theoretical particle physics, gravity and cosmology.\vspace{7pt}
\\ {\bf University of Vienna, Vienna} \hfill {\em 2011 - 2015}
\\ Bachelor of Physics \vspace{7pt}
\\ {\bf Austrian High School Equivalent \emph{Matura}} \hfill {\em 2006 - 2011}
\\ HTL Wels \hfill {graduated with distinction}
\\ School with emphasis on education in technology. Specialisation: Mechatronics

\end{rSection}


\begin{rSection}{Additional Experience}

{\bf Teaching Assistant} \hfill {\em 2019}
\\ Erwin Schrödinger Institute, Vienna, Austria
\\ Course on Supergravity taught by {\bf Antoine Van Proeyen}.\vspace{7pt}
\\{\bf Teaching Assistant} \hfill {\em 2013}
\\ University of Vienna, Vienna, Austria
\\ Basics of Programming for physicists.

\end{rSection}

%--------------------------------------------------------------------------------
%    Publications
%-----------------------------------------------------------------------------------------------

\begin{rSection}{Publications}

{\bf Non-supersymmetric branes}
\\\emph{with:} N. Cribiori, M. Tournoy, A. Van Proeyen and T. Wrase
 \\ JHEP 07 (2020) 189 \hfill arXiv:2004.13110\vspace{7pt}
\\{\bf de Sitter Minima from M theory and String theory}
\\\emph{with:} N. Cribiori, R. Kallosh and A. Linde
 \\ Phys.Rev.D 101 (2020) 4, 046018 \hfill arXiv:1912.02791\vspace{7pt}
\\{\bf Mass Production of IIA and IIB dS Vacua}
\\\emph{with:} N. Cribiori, R. Kallosh and A. Linde
 \\ JHEP 02 (2020) 063 \hfill arXiv:1912.00027\vspace{7pt}
\\{\bf Uplifting Anti-D6-brane}
\\\emph{with:} N. Cribiori, R. Kallosh and T. Wrase
 \\ JHEP 12 (2019) 171 \hfill arXiv:1909.08629\vspace{7pt}
 \newpage
 {\bf Supersymmetric anti-D3-brane action in the Kachru-Kallosh-Linde-Trivedi setup}
\\\emph{with:} N. Cribiori, T. Wrase and Y. Yamada
\\Phys.Rev. D100 (2019) no.6, 066001 \hfill arXiv:1906.07727\vspace{7pt}
\\{\bf Further refining the de Sitter swampland conjecture}
\\\emph{with:} D. Andriot 
\\Fortsch.\ Phys.\  {\bf 67} (2019) no.1-2,  1800105 \hfill arXiv:1811.08889\vspace{7pt}
\\{\bf Scaling limits of dS vacua and the swampland}
\\\emph{with:} A. Banlaki, A. Chowdhury and T. Wrase
\\JHEP 1903 (2019) 065 \hfill arXiv:1811.07880\vspace{7pt}
\\{\bf de Sitter extrema and the swampland}
\\\emph{with:} T. Wrase
\\Fortsch.Phys. 67 (2019) no.1-2, 1800082 \hfill arXiv:1807.09538

\end{rSection}

%----------------------------------------------------------------------------------------
%	Talks
%----------------------------------------------------------------------------------------

\begin{rSection}{Talks}

{\bf Non-Supersymmetric Branes}\hfill{\em 17.11.2020}
\\ Seminar Series on String Phenomenology, Boston, USA (online)\vspace{7pt}
\\{\bf Scaling Limits of dS Vacua}\hfill{\em 27.06.2019}
\\ StringPheno 2019, CERN, Geneva, Switzerland\vspace{7pt}
\\{\bf dS Vacua and Starobinsky Inflation in 4d N=1 Supergravity} \hfill {\em 27.06.2017}
\\ Seminar on Mathematical Physics, University of Vienna, Vienna, Austria

\end{rSection}

%----------------------------------------------------------------------------------------
%	Conferences
%----------------------------------------------------------------------------------------

\begin{rSection}{Conferences and Schools}

{\bf StringPheno21}\hfill {\em 12.-16.7.2021}
\\ Northeastern University, Boston, USA (online)\vspace{7pt}
\\{\bf CERN Winterschool}\hfill {\em 01-05.2.2021}
\\CERN, Geneva, Switzerland (online)\vspace{7pt}
\\{\bf StringPheno20}\hfill {\em 08.-12.6.2020}
\\ Northeastern University, Boston, USA (online)\vspace{7pt}
\\{\bf CERN Winterschool}\hfill {\em 03-07.2.2020}
\\CERN, Geneva, Switzerland\vspace{7pt}
\\{\bf StringPheno19}\hfill {\em 24.-28.6.2019}
\\ CERN, Geneva, Switzerland\vspace{7pt}
\\{\bf CERN Winterschool}\hfill {\em 04-08.2.2019}
\\CERN, Geneva, Switzerland\vspace{7pt}
\\{\bf StringPheno18} \hfill {\em 02.-06.7.2018}
\\University of Warsaw, Warsaw, Poland\vspace{7pt}
\\{\bf CERN Winterschool}\hfill{\em 12-16.2.2018}
\\CERN, Geneva, Switzerland \vspace{7pt}
\\{\bf Laces 2017}\hfill{\em 27.11.-15.12.2017}
\\Galileo Galilei Institute for Theoretical Physics, Florence, Italy

\end{rSection}

%----------------------------------------------------------------------------------------
%	MISC
%----------------------------------------------------------------------------------------
\newpage
\begin{rSection}{Grants}

{\bf Austrian Marshall Plan Fellowship}\hfill {\em 2019}
\\ Travel grant to visit Stanford Institute for Theoretical Physics to collaborate with {\bf Renata Kallosh}.\vspace{7pt}
\\ {\bf DKPI Associate} \hfill {\em 2019 - 2020}
\\ Associate to the \emph{Doktoratskolleg Particles and Interactions} under the supervision of \textbf{Anton Rebhan}.\vspace{7pt}
\\ {\bf ÖAW DOC Stipend}\hfill {\em 2019-2021}
\\ Scholarship for excellent Doctorate candidates of the Austrian Academy of Sciences.

\end{rSection}


\begin{rSection}{Additional Skills}

\begin{tabular}{ @{} >{\bfseries}l @{\hspace{6ex}} l }
Language\ & German (native), English (fluent) \\
Type Setting \ & \LaTeX \\
Mathematics \ & Wolfram Mathematica \\
Programming \ & Python, C, Fortran \\
Code Management \ & Github\\
Office Software \ & Microsoft, Libre, Google\\
Operating Systems \ & Microsoft Windows, Linux (Debian/Ubuntu), Android
\end{tabular}

\end{rSection}




\end{document}