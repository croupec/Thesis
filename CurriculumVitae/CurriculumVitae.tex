\documentclass[a4paper,12pt]{report}
\usepackage{jheppub}
\usepackage[nottoc,numbib]{tocbibind}
\usepackage{amsmath}
\usepackage{amssymb}
\usepackage{graphicx,color,slashed,mathtools,bm,bbm,float,placeins}
\usepackage{bbm} 
\usepackage{ifpdf}
\usepackage{setspace}



\newcommand{\bC}{\mathbb{C}}
\newcommand{\bF}{\mathbb{F}}
\newcommand{\bI}{\mathbb{I}}
\newcommand{\bP}{\mathbb{P}}
\newcommand{\bR}{\mathbb{R}}
\newcommand{\cA}{\mathcal{A}}
\newcommand{\cB}{\mathcal{B}}
\newcommand{\cC}{\mathcal{C}}
\newcommand{\cE}{\mathcal{E}}
\newcommand{\cF}{\mathcal{F}}
\newcommand{\cH}{\mathcal{H}}
\newcommand{\cM}{\mathcal{M}}
\newcommand{\cN}{\mathcal{N}}
\newcommand{\cO}{\mathcal{O}}
\newcommand{\Tr}{\mathrm{Tr\,}}
\newcommand{\ov}{\overline}

\newcommand{\be}{\begin{equation}}
\newcommand{\ee}{\end{equation}}
\newcommand{\bea}{\begin{equation}\begin{aligned}}
\newcommand{\eea}{\end{aligned}\end{equation}}
\newcommand{\ba}{\begin{eqnarray}}
\newcommand{\ea}{\end{eqnarray}}
\newcommand{\nn}{\nonumber}
\newcommand{\lp}{\left(}
\newcommand{\rp}{\right)}
\newcommand{\ls}{\left[}
\newcommand{\rs}{\right]}
\newcommand{\vep}{\varepsilon}
\renewcommand{\d}{\textrm{d}}
\newcommand{\e}{\textrm{e}}
\newcommand{\ep}{\epsilon}
\newcommand{\w}{\wedge}
\renewcommand{\a}{\alpha}
\renewcommand{\b}{\beta}
\newcommand{\N}{\mathcal{N}}
\newcommand{\bZ}{\mathbb{Z}}
\newcommand{\cZ}{\mathcal{Z}}
\newcommand{\cL}{\mathcal{L}}
\renewcommand{\baselinestretch}{1.2}
\def\rmi{{\rm i}}
\def\rme{{\rm e}}
\def\ib{{\bar \imath}}
\def\jb{{\bar \jmath}}
\def\rmre{{\rm Re}}
\def\rmim{{\rm Im}}
\newcommand{\Db}{\overline{D6}}
\newcommand{\A}{\mathcal{A}}
\newcommand{\V}{\mathcal{V}}

\newcommand{\chr}[1]{\textbf{\textcolor{blue}{(#1 --CR)}}}
\newcommand{\tw}[1]{\textbf{\textcolor{red}{(#1 --TW)}}}
\newcommand{\nc}[1]{\textbf{\textcolor{green}{(#1 --NC)}}}
\newcommand{\da}[1]{\textbf{\textcolor{yellow}{(#1 --DA)}}}


%----------------------------------------------------------------------------------------
%	FOR THE CV
%----------------------------------------------------------------------------------------

\usepackage[parfill]{parskip} % Remove paragraph indentation
\usepackage{array} % Required for boldface (\bf and \bfseries) tabular columns
\usepackage{ifthen} % Required for ifthenelse statements
\usepackage{setspace}

% Defines the rSection environment for the large sections within the CV
\newenvironment{rSection}[1]{ % 1 input argument - section name
  \sectionskip
  \MakeUppercase{\bf #1} % Section title
  \sectionlineskip
  \hrule % Horizontal line
  \begin{list}{}{ % List for each individual item in the section
    \setlength{\leftmargin}{1.5em} % Margin within the section
  }
  \item[]
}{
  \end{list}
}

\newenvironment{rSubsection}[4]{ % 4 input arguments - company name, year(s) employed, job title and location
 {\bf #1} \hfill {#2} % Bold company name and date on the right
 \ifthenelse{\equal{#3}{}}{}{ % If the third argument is not specified, don't print the job title and location line
  \\
  {\em #3} \hfill {\em #4} % Italic job title and location
  }\smallskip

  \begin{spacing}{0.75}
  \begin{list}{$\cdot$}{\leftmargin=0em} % \cdot used for bullets, no indentation
   \itemsep -0.5em \vspace{-0.5em} % Compress items in list together for aesthetics
   
  }{
  \end{list}
  \end{spacing}
  \vspace{0.5em} % Some space after the list of bullet points
}

% The below commands define the whitespace after certain things in the document - they can be \smallskip, \medskip or \bigskip
\def\namesize{\huge} % Size of the name at the top of the document
\def\addressskip{\smallskip} % The space between the two address (or phone/email) lines
\def\sectionlineskip{\medskip} % The space above the horizontal line for each section 
\def\nameskip{\bigskip} % The space after your name at the top
\def\sectionskip{\medskip} % The space after the heading section

%----------------------------------------------------------------------------------------
%
%----------------------------------------------------------------------------------------


\title{Flux Compactifications, dS Vacua and the Swampland}
\author{Christoph Oliver Roupec MSc}
\begin{document}

\center
\section*{Christoph Oliver ROUPEC}
Institute for Theoretical Physics, Vienna University of Technology\\
Wiedner Hauptstraße 8-10/136, A-1040 Vienna, Austria\\
Christoph.Roupec@gmail.com\\
+43 664 3826128

\begin{rSection}{Personal Information}
      {\bf Date of Birth} \hfill { 31st of July 1991}
      \\{\bf Place of Birth} \hfill { Vöcklabruck, Austria}
      \\{\bf Nationality} \hfill { Austria}
      \\{\bf Address} \hfill {Breitenseer Straße 39a, T23-24, A-1140, Vienna, Austria}

\end{rSection}

\begin{rSection}{Experience}

      {\bf Vienna University of Technology, Institute for Theoretical Physics, Vienna} \hfill {\em 2017 - 2022} 
      \\ Doctorate in Theoretical Physics under the supervision of {\bf Timm Wrase}.
      \\ Thesis Title: \emph{Flux Compactifications, dS Vacua and the Swampland}\vspace{7pt}
      \\{\bf Stanford Institute for Theoretical Physics} \hfill {\em August - December 2019}
      \\ Visiting Scientist working with {\bf Renata Kallosh} and {\bf Andrei Linde}.
      %\\ Uplifting $\overline{D6}$-branes
      \vspace{7pt}
      \\{\bf University of Vienna, Vienna} \hfill {\em 2015 - 2017} 
      \\ Master of Physics\hfill {graduated with distinction }
      \\ Emphasis on theoretical particle physics, gravity and cosmology.\vspace{7pt}
      \\ {\bf University of Vienna, Vienna} \hfill {\em 2011 - 2015}
      \\ Bachelor of Physics \vspace{7pt}
      \\ {\bf Austrian High School Equivalent \emph{Matura}} \hfill {\em 2006 - 2011}
      \\ HTL Wels \hfill {graduated with distinction}
      \\ School with emphasis on education in technology. Specialisation: Mechatronics
      
      \end{rSection}

      \begin{rSection}{Additional Experience}

            {\bf Teaching Assistant} \hfill {\em 2019}
            \\ Erwin Schrödinger Institute, Vienna, Austria
            \\ Course on Supergravity taught by {\bf Antoine Van Proeyen}.\vspace{7pt}
            \\{\bf Teaching Assistant} \hfill {\em 2013}
            \\ University of Vienna, Vienna, Austria
            \\ Basics of Programming for physicists.
            
            \end{rSection}
            
            %--------------------------------------------------------------------------------
            %    Publications
            %-----------------------------------------------------------------------------------------------
            \newpage
            \begin{rSection}{Publications}
            
            {\bf Non-supersymmetric branes}
            \\\emph{with:} N. Cribiori, M. Tournoy, A. Van Proeyen and T. Wrase
             \\ JHEP 07 (2020) 189 \hfill arXiv:2004.13110\vspace{7pt}
            \\{\bf de Sitter Minima from M theory and String theory}
            \\\emph{with:} N. Cribiori, R. Kallosh and A. Linde
             \\ Phys.Rev.D 101 (2020) 4, 046018 \hfill arXiv:1912.02791\vspace{7pt}
            \\{\bf Mass Production of IIA and IIB dS Vacua}
            \\\emph{with:} N. Cribiori, R. Kallosh and A. Linde
             \\ JHEP 02 (2020) 063 \hfill arXiv:1912.00027\vspace{7pt}
            \\{\bf Uplifting Anti-D6-brane}
            \\\emph{with:} N. Cribiori, R. Kallosh and T. Wrase
             \\ JHEP 12 (2019) 171 \hfill arXiv:1909.08629\vspace{7pt}
             \\{\bf Supersymmetric anti-D3-brane action in the Kachru-Kallosh-Linde-Trivedi setup}
            \\\emph{with:} N. Cribiori, T. Wrase and Y. Yamada
            \\Phys.Rev. D100 (2019) no.6, 066001 \hfill arXiv:1906.07727\vspace{7pt}
            \\{\bf Further refining the de Sitter swampland conjecture}
            \\\emph{with:} D. Andriot 
            \\Fortsch.\ Phys.\  {\bf 67} (2019) no.1-2,  1800105 \hfill arXiv:1811.08889\vspace{7pt}
            \\{\bf Scaling limits of dS vacua and the swampland}
            \\\emph{with:} A. Banlaki, A. Chowdhury and T. Wrase
            \\JHEP 1903 (2019) 065 \hfill arXiv:1811.07880\vspace{7pt}
            \\{\bf de Sitter extrema and the swampland}
            \\\emph{with:} T. Wrase
            \\Fortsch.Phys. 67 (2019) no.1-2, 1800082 \hfill arXiv:1807.09538
            
            \end{rSection}
            
            %----------------------------------------------------------------------------------------
            %	Talks
            %----------------------------------------------------------------------------------------
            
            \begin{rSection}{Talks}
            
            {\bf Non-Supersymmetric Branes}\hfill{\em 17.11.2020}
            \\ Seminar Series on String Phenomenology, Boston, USA (online)\vspace{7pt}
            \\{\bf Scaling Limits of dS Vacua}\hfill{\em 27.06.2019}
            \\ StringPheno 2019, CERN, Geneva, Switzerland\vspace{7pt}
            \newpage
            {\bf dS Vacua and Starobinsky Inflation in 4d N=1 Supergravity} \hfill {\em 27.06.2017}
            \\ Seminar on Mathematical Physics, University of Vienna, Vienna, Austria
            
            \end{rSection}
            
            %----------------------------------------------------------------------------------------
            %	Conferences
            %----------------------------------------------------------------------------------------
            
            \begin{rSection}{Conferences and Schools}
            
            {\bf StringPheno21}\hfill {\em 12.-16.7.2021}
            \\ Northeastern University, Boston, USA (online)\vspace{7pt}
            \\{\bf CERN Winterschool}\hfill {\em 01-05.2.2021}
            \\CERN, Geneva, Switzerland (online)\vspace{7pt}
            \\{\bf StringPheno20}\hfill {\em 08.-12.6.2020}
            \\ Northeastern University, Boston, USA (online)\vspace{7pt}
            \\{\bf CERN Winterschool}\hfill {\em 03-07.2.2020}
            \\CERN, Geneva, Switzerland\vspace{7pt}
            \\{\bf StringPheno19}\hfill {\em 24.-28.6.2019}
            \\ CERN, Geneva, Switzerland\vspace{7pt}
            \\{\bf CERN Winterschool}\hfill {\em 04-08.2.2019}
            \\CERN, Geneva, Switzerland\vspace{7pt}
            \\{\bf StringPheno18} \hfill {\em 02.-06.7.2018}
            \\University of Warsaw, Warsaw, Poland\vspace{7pt}
            \\{\bf CERN Winterschool}\hfill{\em 12-16.2.2018}
            \\CERN, Geneva, Switzerland \vspace{7pt}
            \\{\bf Laces 2017}\hfill{\em 27.11.-15.12.2017}
            \\Galileo Galilei Institute for Theoretical Physics, Florence, Italy
            
            \end{rSection}
            
            %----------------------------------------------------------------------------------------
            %	MISC
            %----------------------------------------------------------------------------------------
            \begin{rSection}{Grants}
            
            {\bf Austrian Marshall Plan Fellowship}\hfill {\em 2019}
            \\ Travel grant to visit Stanford Institute for Theoretical Physics to collaborate with {\bf Renata Kallosh}.\vspace{7pt}
            \\ {\bf DKPI Associate} \hfill {\em 2019 - 2020}
            \\ Associate to the \emph{Doktoratskolleg Particles and Interactions} under the supervision of \textbf{Anton Rebhan}.\vspace{7pt}
            \\ {\bf ÖAW DOC Stipend}\hfill {\em 2019-2021}
            \\ Scholarship for excellent Doctorate candidates of the Austrian Academy of Sciences.
            
            \end{rSection}
            
            
            \begin{rSection}{Additional Skills}
            
            \begin{tabular}{ @{} >{\bfseries}l @{\hspace{6ex}} l }
            Language\ & German (native), English (fluent) \\
            Type Setting \ & \LaTeX \\
            Mathematics \ & Wolfram Mathematica \\
            Programming \ & Python, C, Fortran \\
            Code Management \ & Github\\
            Office Software \ & Microsoft, Libre, Google\\
            Operating Systems \ & Microsoft Windows, Linux (Debian/Ubuntu), Android
            \end{tabular}
            
            \end{rSection}
            

\end{document}